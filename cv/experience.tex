%-------------------------------------------------------------------------------
%	SECTION TITLE
%-------------------------------------------------------------------------------
\cvsection{Work Experience}


%-------------------------------------------------------------------------------
%	CONTENT
%-------------------------------------------------------------------------------
\begin{cventries}
	
	
	
	
	\cventry
	{Research Intern} % Job title
	{Toyota Research Institute} % Organisation
	{Cambridge, MA} % Location
	{May 2024 - Aug 2024} % Date(s)
	{
		\begin{cvitems} % Description(s) of tasks/responsibilities
			\item {Incoming Summer Intern at Toyota Research Institute, Robotics Department, Dynamics and Simulation branch.}
			\item {Will assist the development and the parallelization of Robotics Model-Based Design and Verification Toolbox Drake, co-developed by TRI and MIT Locomotion Lab.}
			\item{Will assist robotics imitation and reinforcement learning enabled by simulation.}
		\end{cvitems}
	}
	
	
	
	\cventry
	{Graduate Teaching Assistant} % Job title
	{UW-Madison} % Organisation
	{Madison, WI} % Location
	{Jun 2022 - Present} % Date(s)
	{
		\begin{cvitems} % Description(s) of tasks/responsibilities
			\item {\textbf{ME/ECE/COMP SCI 759}: High Performance Computing - Parallel Computing Programming, CUDA, OpenMP, MPI; Parallel Program Optimization, GPU and Accelerator Architectures.}
			\item {\textbf{ME 451}: Multibody Dynamics - Kinematics and Dynamics; Forward and Inverse Dynamics; Friction and Contact.}
		\end{cvitems}
	}
	
	
	
  \cventry
	{Graduate Research Assistant} % Job title
	{Simulation-Based Engineering Lab at UW-Madison} % Organisation
	{Madison, WI} % Location
	{Jan 2021 - Present} % Date(s)
	{
		\begin{cvitems} % Description(s) of tasks/responsibilities
			\item {Advisor: Professor Dan Negrut.}
			\item {Developed autonomous vehicle coordination and simulation, leveraged simulation fidelity and real-time performance for Human-In-The-Loop, Hardware-In-The-Loop and Software-In-The-Loop applications. Head developer of chrono::HIL, a plug-in of Project Chrono (\url{https://projectchrono.org/}) to provide human-in-the-loop and real-time simulation support for traffic scenarios and vehicle dynamics. chrono:HIL provides flexible simulator hardware coupling capabilities, distributed simulation support, soft real-time simulation support, and multiple vehicle dynamic models. Integrated Chrono::HIL with National Advanced Driving Simulator (NADS) at the University of Iowa.} 
			\item {Integrated sensor (lidar/radar/camera sensor) simulation using chrono::sensor into traffic scenarios to assist the development of autonomous vehicle control and sensor fusion. Provided simulation support for human-factor research conducted by Cognitive Systems Laboratory at UW-Madison. Program is based on C++ for high performance, leveraging DDS, CUDA, and Socket Programming for distributed computing. Funded by National Science Foundation OAC2209791.}
			\item {Involved in the development of Gym-Chrono, a Deep Reinforcement Learning environment and pipeline using Project Chrono to enable research on the transfer of learned policy from simulation to reality. Validation of Software-In-The-Loop training using LEGO-SLAM algorithm.}
			\item{
			Robotics driver software suite built upon ROS2 and FreeRTOS for SBEL physical validation robots. Integrated Lidar, IMU, and GPS sensors for real-time data acquisition and processing using C++, python, and ROS2. software-In-The-Loop and hardware-in-the-loop simulation using DDS, TCP/UDP socket for control and SLAM (Simultaneous Localization and Mapping) validations. }
			\item {Extraterrestrial rover and robot mechanical component simulation. Applications/development/validation of SCM (Soil Contact Model), SPH (Smoothed Particle Hydrodynamics), and DEM (Discrete Element Method) deformable terrain. Developed the VIPER lunar rover model and the Curiosity Mars rover model in the chrono::robot module. Integrated sensor simulation support to provide Lidar/Radar perception data in harsh lunar environment. The program is based on CUDA and Nvidia Optix for GPU-Based physics solver and ray-tracing. Funded under NASA project to support 2023 VIPER lunar mission.}
		\end{cvitems}
	}


		
	
%--------------------------------------------------------

  \cventry
	{Undergraduate Research Assistant} % Job title
	{Simulation-Based Engineering Lab at UW-Madison} % Organisation
	{Madison, WI} % Location
	{Jun 2020 - Dec 2020} % Date(s)
	{
		\begin{cvitems} % Description(s) of tasks/responsibilities
			\item {Advisor: Professor Dan Negrut.}
			\item {Developed and validated chrono::granular (later renamed chrono::gpu), a CUDA solver for granular dynamics. chrono::granular can be used to simulate monodisperse granular material; applications include granular material properties testing and deformable terrain for off-road vehicle research; simulator solved problem with more than one billion degrees of freedom on one V100 NVIDIA GPU.}
			\item {Developed chrono::synchrono, a Project Chrono plugin that supports the space and time-coherent simulation of multiple vehicles in Chrono. Developed MPI and DDS interfaces of chrono::vehicle; utilized of parallel computing for achieving real-time performance.}
		\end{cvitems}
	}


% --------------------------------------------------------

  \cventry
	{Undergraduate Research Assistant} % Job title
	{Human Computer Interaction Lab at UW-Madison} % Organisation
	{Madison, WI} % Location
	{Sep 2019 - May 2020} % Date(s)
	{
	  \begin{cvitems} % Description(s) of tasks/responsibilities
		\item {Advisor: Professor Bilge Mutlu.}
		\item {Developed a QR Marker object tracking program based on OpenCV in C++. The program helps educational robots to identify objects and their movements in order to facilitate human-computer interaction.}
		\item {Designed and developed of simulation environment for robot localization algorithm using ROS2. The simulation environment allows a Turtlebot model to follow certain trajectories in an indoor environment relying purely on QR codes identified by the machine learning algorithm.}
		\item {Created of the CAD models for robot's parts  using Solidworks and 3D printing software.}
	  \end{cvitems}
}


%---------------------------------------------------------
  \cventry
    {Software Engineering Intern \& Embedded System Engineering Intern} % Job title
    {Alstom} % Organisation
    {Melbourne, FL} % Location
    {May 2019 - Aug 2019} % Date(s)
    {
      \begin{cvitems} % Description(s) of tasks/responsibilities
        \item {Cooperated with Alstom's System Validation Team to perform system tests and review code (primarily in C++ and Python) on Alstom DAU (Data Acquisition Unit), a critical wayside component of the Alstom's Automatic Railway Signaling System; Debugging lower-level program, scanning and hacking the Apache server installed to search for possible bugs which may lead to the fatal crash of the system.}
        \item {Developed a C++ testing program for Alstom's Wayside Linux-Based Core ACE board to meet Hardware Serial Test Specifications including multi-CPU communication, I2C, UART, SPI, onboard GPIO connection, Watchdog Timer, and other hardware checks. The testing program includes both lower-level hardware programming and higher-level algorithm programming.}
      \end{cvitems}
    }

%---------------------------------------------------------
\end{cventries}
